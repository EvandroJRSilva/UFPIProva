%===============================================================================
% Cabeçalho e Rodapé -----------------------------------------------------------
%===============================================================================

\firstpagefooter{\professor}{}{\disciplina}
\firstpagefootrule
\runningfooter{\professor}{\disciplina}{Pág. \thepage\ de \numpages}
\runningfootrule


%===============================================================================
% Informações sobre a Avaliação ------------------------------------------------
%===============================================================================
\LogoUFPI{\includegraphics[width=\linewidth]{imagens/logos/brasao-ufpi.png}}

% Insira o nome do curso a seguir. Ex.: Bacharelado em Direito
\NomeCurso{Nome do Curso}

% Insira o nome do(a) Professor(a). Ex.: Fulano da Silva
\NomeProfessor{Nome do(a) Professor(a)}

% Insira o nome da Disciplina. Ex.: Citologia
\NomeDisciplina{Nome da Disciplina}

% Insira o Código da Turma. Exemplo: A
\CodTurma{A}

% Insira o semestre. Exemplo: 2024.2
\Semestre{202X.X}

%===============================================================================
% Instruções para a Avaliação --------------------------------------------------
%===============================================================================

% OBS.: deixe em branco as que não quiser colocar. Mas caso queira cinco ou mais instruções, descomente \instcinco{} em diante, e utilize novos comandos caso seja necessário. No arquivo UninassauProva.sty procure pelos trechos em que os comandos são criados, como também o trecho em que os comandos são mostrados na prova.

\instum{A avaliação é individual e não é pesquisada.}
 
\instdois{Preencha o cabeçalho da folha pergunta com seus dados.}

\insttres{O preenchimento das respostas deve ser feito utilizado caneta (preta ou azul).}

%\instquatro{}

%\instcinco{}

%\instseis{}
